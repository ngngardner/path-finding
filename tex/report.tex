\documentclass[sigplan,screen]{acmart}

%%
%% \BibTeX command to typeset BibTeX logo in the docs
\AtBeginDocument{%
  \providecommand\BibTeX{{%
    \normalfont B\kern-0.5em{\scshape i\kern-0.25em b}\kern-0.8em\TeX}}}

%% Rights management information.  This information is sent to you % when you
%complete the rights form.  These commands have SAMPLE % values in them; it is
%your responsibility as an author to replace % the commands and values with
%those provided to you when you % complete the rights form.
\setcopyright{acmcopyright}
\copyrightyear{2018}
\acmYear{2018} \acmDOI{10.1145/1122445.1122456}

\begin{document}
\title{Agent Path Planning}
\author{Noah Gardner}
\authornote{All authors contributed equally to this research.}
\email{ngardn10@students.kennesaw.edu}
\orcid{0000-0001-5900-9841}
\affiliation{%
    \institution{College of Computing and Software Engineering }
    \streetaddress{1100 South Marietta}
    \city{Marietta}
    \state{Georgia}
    \country{USA}
    \postcode{30060}
}

\author{Alan Norkham}
\email{@students.kennesaw.edu}
\affiliation{%
    \institution{College of Computing and Software Engineering }
    \streetaddress{1100 South Marietta}
    \city{Marietta}
    \state{Georgia}
    \country{USA}
    \postcode{30060}
}

\author{Mikulas Chalupa}
\email{@students.kennesaw.edu}
\affiliation{%
    \institution{College of Computing and Software Engineering }
    \streetaddress{1100 South Marietta}
    \city{Marietta}
    \state{Georgia}
    \country{USA}
    \postcode{30060}
}
\renewcommand{\shortauthors}{Gardner et al.}

\begin{abstract}
\end{abstract}
\keywords{path finding}
\maketitle

\section{Introduction}
The problem of path planning is a difficult problem for mobile robots. A
practical example can be seen in the robots commonly employed in warehouses:
they must navigate to pick up goods and move them to certain locations.
Therefore, the robot needs a method of moving from an initial location in the
warehouse to a final location repeatedly.

Environmental factors such as collision avoidance and the fastest path taken is
very important.

\section{Dataset}
Our dataset will be a custom one, which consists of a 'grid-world' environment.
It will contain our agent, some obstacles, and a goal. We would like to test
with different starting and goal locations as well as random initialization of
obstacles. Obstacles may completely block the agent from moving into the cell
with the obstacle, or the obstacle could have an increased or decreased cost of
moving into the cell, for example, a sand pit or road.

\section{Model}
We would like to apply an RNN or LSTM to this problem. Given the research we
collected, these models should be appropriate for this kind of problem.

\section{Methodology}
Our goal for this project will to be to demonstrate a learned agent that can
pathfind effectively from the starting location to the goal location.

\subsection{Evaluation}
We will evaluate our learned path-finding algorithm by comparing it to commonly
used path-finding algorithms such as Dijkstra's algorithm and A* algorithm.

\bibliographystyle{ACM-Reference-Format}
\bibliography{report}
\end{document}
